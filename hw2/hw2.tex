\documentclass[10pt,letterpaper]{report}
\usepackage[utf8]{inputenc}
\usepackage[margin=0.65in]{geometry}
\usepackage{amsmath}
\usepackage{amsfonts}
\usepackage{amssymb}
\usepackage{fancyvrb}
\usepackage{appendix}
\usepackage{listings}
\usepackage{color}


\author{Marc Evangelista}
\title{Problem Set 2}
\begin{document}
\begin{flushright}
Marc Evangelista \\
MATH 4175 \\ 
CRN: 94509 \\
Problem Set 2 \\
Due: 2013/09/09
\end{flushright}
	All indexing in this document starts at 0 unless otherwise specified.
\begin{enumerate}
	\item The following was encrypted by the Vigenere method. Decrypt it. Carefully document your work–in general, more details are better than fewer details.
	
		\begin{verbatim}
		XKJUROWMLLPXWZNPIMBVBQJCNOWXPCCHHVVFVSLLFVXHAZITYXOHULX 
		QOJAXELXZXMYJAQFSTSRULHHUCDSKBXKNJQIDALLPQSLLUHIAQFPBPC
		IDSVCIHWHWEWTHBTXRLJNRSNCIHUVFFUXVOUKJLJSWMAQFVJWJSDYLJ
		OGJXDBOXAJULTUCPZMPLIWMLUBZXVOODYBAFDSKXGQFADSHXNXEHSAR
		UOJAQFPFKNDHSAAFVULLUWTAQFRUPWJRSZXGPFUTJQIYNRXNYNTWMHC
		\end{verbatim}

	First, we find possible repeated strings within the ciphertext. We start off small, 
	using a repetition string size of 3. Using a program to search every possible 
	3-character string in the cipher (Listing B.1), we find that \verb|AQF| is repeated 5 times.
	\par
	Next, we analyze the positions of those repetitions. The indices at which \verb|AQF|
	appear are at [68, 103, 153, 223, 243]. So we will calculate the difference between
	these positions, then find a common factor. These common factors will be our possible
	key sizes.
	\begin{align*}
			103 - 68 &= 35 \\
			153 - 103 &= 50 \\
			223 - 153 &= 70 \\
			243 - 223 &= 20		
	\end{align*}
	Clearly, the GCD of these differences is 5, so we will choose that as our keysize.
	Now we chop the ciphertext into rows of 5 characters each, so we will have 5 columns.
	See Listing B.2 for the full listing.
	
	\begin{verbatim}
	XKJUR
	OWMLL
	PXWZN
	PIMBV
	BQJCN
	...
	RXNYN
	TWMHC
	\end{verbatim}
	
	Each column corresponds to a simple Caesar shift, so we can treat each column as its
	own individual cipher to solve. So we can apply frequency analysis techniques on each
	individual column, and make educated guesses on the amount of shift used.
	\par 
	First, we will count the letter frequency for the first column. See Listing B.3.
	\begin{verbatim}
	Column 0: XOPPBOCFFZOQEMFUCXIPUFIIETNIFUSFSOBUPIBOFGSEUFDFUFJGJRT
	B: 3 (5.45%)
	C: 2 (3.64%)
	D: 1 (1.82%)
	E: 3 (5.45%)
	F: 10 (18.18%)
	G: 2 (3.64%)
	I: 5 (9.09%)
	J: 2 (3.64%)
	M: 1 (1.82%)
	N: 1 (1.82%)
	O: 5 (9.09%)
	P: 4 (7.27%)
	Q: 1 (1.82%)
	R: 1 (1.82%)
	S: 3 (5.45%)
	T: 2 (3.64%)
	U: 6 (10.91%)
	X: 2 (3.64%)
	Z: 1 (1.82%)
	\end{verbatim}
	
	Since F is the most common letter, we will assume that it corresponds to the plaintext
	letter E, the most common letter in English. The distance from F to E can be 
	calculated using the formula  $(char_{plain} - char_{cipher})\ mod\ 26$. In other words, that distance
	mod 26 is the amount of letters to shift from F to E. So,
		\begin{align*}
		('E' - 'F')\ mod\ 26 &= (4 - 5) mod 26 \\
		&= -1 mod 26 \\
		&= 25 \\
		\end{align*}

	So, shifting F by 25 will give us E. Applying this logic to each column gives us the 
	following table:
	
	\begin{center}
	\begin{tabular}{ | c | c | c | c | c  c  c  c |}
	\hline
	& & & & Distance To & & &\\ \hline 
	Col. & Most Freq. Char & Freq & Index & E & T & A & O \\ \hline
	0 & F & 18\% & 5 & 25 & 14 & 21 & 9 \\ 
	1 & H & 14\% & 7 & 23 & 12 & 19 & 7 \\
	2 & S & 16\% & 18 & 12 & 1 & 8 & 22 \\
	3 & L & 18\% & 11 & 19 & 8 & 15 & 3 \\
	4 & X & 12\% & 23 & 7 & 22 & 3 & 17 \\
	\hline
	\end{tabular}
	\end{center}
	Included are distances to other most common letters. Now we assume that all of the
	most frequent letters were decrypted from E. So shifting each column by its
	corresponding distance to E as seen in the table, we get:
	
	\begin{verbatim}
	Column 0 shifted by 25: 
	wnooanbeeynpdletbwhotehhdsmhetrernatohanefrdteceteifiqs
	Column 1 shifted by 23: 
	htufntessfelivpiahanemaetuoerhtsadliwtwaaneelmestoomnut
	Column 2 shifted by 12: 
	vyiyvjtejfgvjvftezmeuneifdegjvyvkvjfyyjkerjevregfgeruzy
	Column 3 shifted by 19: 
	nesuvioearetstladceetioaaegooetpeqtnieoudtgttdtetisnrra
	Column 4 shifted by 07: 
	ysucujcsheeeexybixssxjjdiqjmvqxqqkqjsbvhekeyxuhsxdeauuj
	\end{verbatim}
	or alternatively, in actual columns (Listing B.4):
	\begin{verbatim}
	whvny
	ntyes
	ouisu
	ofyuc
	...
	quzru
	styaj
	\end{verbatim}
	
	Now we can try exhausting the search with each possible shift, giving us $25^5$
	possibilities to try. We would ideally start with the shifts from the more
	common letters in the table. Instead, we can analyze the text from the E shifts that
	we just tried.
	\par
	The first 6 characters, \verb|whvnyv|, look like the words 'when in' or 'when on'.
	So we will assume $w$, $h$, and $n$, are all correct, along with their corresponding
	column. That leaves two columns left to decipher: 2 and 4.
	\par
	In Column 2, our original ciphertext \verb|J| will need to be shifted to \verb|e|,
	because 'when' occurs in both our suspected plaintext phrases. Using our earlier
	formula:
	\begin{align*}
		(char_{plain} - char_{cipher})\ mod\ 26 &= ('E' - 'J')\ mod\ 26 \\
		 &= (4 - 9) mod 26 \\
		&= -5\ mod\ 26 \\
		&= 21 \\
	\end{align*}
	We will shift Column 2 by 21. \\
	Similarly for Column 4, we shift \verb|R| to both \verb|i| and \verb|o| as in 'in' and 'on' respectively. The resulting shifts are 17 and 23 respectively. So, we get
	\par
	\begin{verbatim}
	R -> i
	Column 0 shifted by 25: wnooanbeeynpdletbwhotehhdsmhetrernatohanefrdteceteifiqs
	Column 1 shifted by 23: htufntessfelivpiahanemaetuoerhtsadliwtwaaneelmestoomnut
	Column 2 shifted by 21: ehrhescnsopeseocnivndwnromnpsehetesohhstnasneanpopnadih
	Column 3 shifted by 19: nesuvioearetstladceetioaaegooetpeqtnieoudtgttdtetisnrra
	Column 4 shifted by 17: icemetmcroooohilshcchttnsatwfahaauatclfrouoiherchnokeet
	
	R -> o
	Column 0 shifted by 25: wnooanbeeynpdletbwhotehhdsmhetrernatohanefrdteceteifiqs
	Column 1 shifted by 23: htufntessfelivpiahanemaetuoerhtsadliwtwaaneelmestoomnut
	Column 2 shifted by 21: ehrhescnsopeseocnivndwnromnpsehetesohhstnasneanpopnadih
	Column 3 shifted by 19: nesuvioearetstladceetioaaegooetpeqtnieoudtgttdtetisnrra
	Column 4 shifted by 23: oikskzsixuuuunoryniinzztygzclgnggagzirlxuauonkxintuqkkz
	\end{verbatim}
	The columns above are written horizontally for space.\\
	We can clearly see that shifting Column 3 by 19 gives us more legible text. So
	we assume we shift Columns 0 to 4 by (25, 23, 21, 19, 17) respectively. Now we
	just have to concatenate each row of each Column for our final plaintext:
	
	\begin{verbatim}
	wheninthecourseofhumaneventsitbecomesnecessaryforonepeo
	pletodissolvethepoliticalbandswhichhaveconnectedthemwit
	hanotherandtoassumeamongthepowersoftheearththeseparatea
	ndequalstationtowhichthelawsofnatureandofnaturesgodenti
	tlethemadecentrespecttotheopinionsofmankindrequiresthat
	\end{verbatim}

	We were fortunate that 3 of 5 guesses were correct, allowing us to guess some words.
	In the case that our initial guesses were not correct and could not guess at
	words or patterns, we would try exhausting all the shifts, starting with the
	more frequent letter mappings like the chart.
	\\
	
	
	
	\item Suppose you have a language with only the three letters $a$, $b$, $c$, and
	they occur with frequencies 0.7, 0.2, and 0.1, respectively. The following ciphertext
	was encrypted by the Vigenere method (shifts are mod 3 instead of mod 26):
	
	\begin{center}
	\verb|ABCBABBBAC|
	\end{center}
	
	You are told that the key length is 1, 2, or 3. Show that the key length is
	probably 2, and determine the most probable key.
	Again, document your work.
	\par
	We will use the Friedman test to determine the probably key length. This is modeled
	by:
	$$ \frac{\kappa_p - \kappa_r}{\kappa_o - \kappa_r} $$
	where $\kappa_p$ is the index of coincidence of the language, $\kappa_r = \frac{1}{length_{alphabet}}$, $\kappa_o$ is theapproximate IC using a sample.
	
	First, we determine the index of coincidence $\kappa_p$ for this language:
	\begin{align*}
	IC = \kappa_p &= \sum\limits_{j=0}^{2} p_j^2 \\
	&= 0.7^2 + 0.2^2 + 0.1^2 \\
	&= 0.54
	\end{align*}

	
	Next, want to find $\kappa_o$. This is modeled as:
	\begin{align*}
	\kappa_o &= \sum\limits_{j=0}^{2} \frac{f_j(f_j - 1)}{N(N-1)}
	\end{align*}
	where $f_j$ is the frequency of the $j$th character in the sample, and $N$ is the
	sample size. We know $N=10$, but we also need to count the frequencies of each
	character. Following is a frequency count of our sample:
	\begin{verbatim}
	$ frequency.py ABCBABBBAC
	A: 3 (30.00%)
	B: 5 (50.00%)
	C: 2 (20.00%)
	\end{verbatim}
	
	Now we can plug in our values to find $\kappa_o$.
	\begin{align*}
	\kappa_o &= \sum\limits_{j=0}^{2} \frac{f_j(f_j - 1)}{N(N-1)} \\
	&= \frac{3(3-1)}{10(10-1)} + \frac{5(5-1)}{10(10-1)} + \frac{2(2-1)}{10(10-1)} \\
	&= \frac{3(2)+5(4)+2(1)}{90} \\ 
	&= 0.311
	\end{align*}
	
	Additionally, $\kappa_r = \frac{1}{3} = 0.333$. Plugging them back into the formula for the Friedman test,
	\begin{align*}
	\frac{\kappa_p - \kappa_r}{\kappa_o - \kappa_r} &= \frac{0.54 - 0.333}{0.311-0.333}
	\\ &= -9.409
	\end{align*}
	
	$-9.409$ is obviously not a valid keylength. This may be due to our small sample size.
	\par
	Instead, we can try the Kasiski examination. Here, we look at repeated substrings
	within our ciphertext. Following is a table of repeated strings, their positions, and
	differences in positions.
	
	\begin{center}
	\begin{tabular}{ | c | c | l |}
	\hline
	Substring & Position & Difference   \\ \hline
	AB & 0, 4 & $4-0=4$ \\
	BA & 3, 7 & $7-3=4$ \\
	BB & 5, 6 & $6-5=1$ \\
	\hline
	\end{tabular}
	\end{center}
	
	\verb|BB|'s repetition probably less significance because it
	has a position difference of 1, which is the GCD for any prime number. Instead, we
	look at the other substrings.
	Looking at the differences of strings \verb|AB| and \verb|BA|, we see that the GCD is
	4. However, 4 is not within our probable keylengths, so we use the next greatest
	common denominator, 2, as our most probably keylength.
	
	Now we can split the ciphertext into columns, then find the shift of each column with
	frequency analysis.
	So,
	\begin{verbatim}
	AB
	CB
	AB
	BB
	AC
	
	$ frequency.py ACABA
	A: 3 (60.00%)
	B: 1 (20.00%)
	C: 1 (20.00%)
	
	$ frequency.py BBBBC
	B: 4 (80.00%)
	C: 1 (20.00%)
	\end{verbatim}
	
	Again, we will assume the most frequent letter in the column corresponds to 
	the most frequent letter found in the language (in this case, $a$).
	
	\begin{center}
	\begin{tabular}{ | c | c | c | c |}
	\hline
	Col & Freq. Char & Distance from $a$ & Key Char \\ \hline
	0 & $A$ & 0  & a\\
	1 & $B$ & 1 & b\\
	\hline
	\end{tabular}
	
	\end{center}
	Note that the distance column is different from the one in Problem 1.
	\par	
	 This is 
	distance from the frequent letter, not to.
	Concatenating the Key Char column gives us the most probable key $ab$.
	
	
	
	\item The ciphertext \verb|GEXZDS| was encrypted by a Hill cipher with a $2 \times 2$ 		encryption matrix M. The plaintext is \verb|solved|. 
	Find the encryption matrix M.
	\par
	First, we split both the plaintext and corresponding ciphertext into chunks of 2,
	because our key is $2 \times 2$. So we have \verb|so, lv, ed| and \verb|GE, XZ, DS|
	respectively. If we let $K = \begin{pmatrix}
	a & b \\ c & d
	\end{pmatrix}$ be our encryption key, we can write this cipher as
	$$	
	K	
	\begin{pmatrix}
	's' \\ 'o'
	\end{pmatrix} = 
	\begin{pmatrix}
	'G' \\ 'E'
	\end{pmatrix}
	$$
	So the key multiplied by a vector with our two-character plaintext chunk gives us the
	ciphertext chunk. If we use square matrices, we can then use the inverse of our
	plaintext matrix to solve for $K$. Let $P$ be our plaintext matrix and $C$ be our
	ciphertext matrix. So,
	\begin{align*}
	KP &= C \\
	KPP^{-1} &= CP^{-1} \\
	K &= CP^{-1}
	\end{align*}

	We will use the pairs \verb|lv| and \verb|ed| because they are smaller integers to
	work with. So,
	\begin{align*}
		K \begin{pmatrix} 'l' & 'e' \\ 'v' & 'd' \end{pmatrix} &=
	\begin{pmatrix} 'X' & 'D' \\ 'Z' & 'S'	\end{pmatrix}
	\\
	K \begin{pmatrix} 11 & 4 \\ 21 & 3	\end{pmatrix} &=
	\begin{pmatrix} 23 & 3 \\ 25 & 18	\end{pmatrix}
	\end{align*}
	Now we find the inverse of $\begin{pmatrix} 11 & 4 \\ 21 & 3 \end{pmatrix}$.
	\begin{align*}
	\begin{pmatrix} 11 & 4 \\ 21 & 3	\end{pmatrix}^{-1} &= 
	\frac{1}{-51}\begin{pmatrix}
	3 & -4 \\ -21 & 11 
	\end{pmatrix} (mod\ 26)	 \\
	&= \begin{pmatrix}
	3 & 22 \\ 5 & 11
	\end{pmatrix} (mod\ 26)	
	\end{align*}
	
	Now plug it back in to solve for K.
		\begin{align*}
	K \begin{pmatrix} 11 & 4 \\ 21 & 3	\end{pmatrix} &=
	\begin{pmatrix} 23 & 3 \\ 25 & 18	\end{pmatrix} \\
		K \begin{pmatrix} 11 & 4 \\ 21 & 3	\end{pmatrix}
		\begin{pmatrix} 11 & 4 \\ 21 & 3	\end{pmatrix}^{-1}		 &=
	\begin{pmatrix} 23 & 3 \\ 25 & 18	\end{pmatrix} \begin{pmatrix} 11 & 4 \\ 21 & 3	\end{pmatrix} 
	\\ K &= \begin{pmatrix}
	6 & 19 \\ 9 & 20
	\end{pmatrix}
	\end{align*}

	$K = \begin{pmatrix}
	6 & 19 \\ 9 & 20
	\end{pmatrix}$ is our encryption key.
		
\end{enumerate}

\begin{appendices}
\chapter{Code Listings}
\lstinputlisting[language=Python, 
				breaklines=true,
				numbers=left,
				caption=problem1.py,
				label={problem1.py}]{problem1.py}
\lstinputlisting[language=Python, 
				breaklines=true,
				numbers=left,
				caption=letter.py,
				label={letter.py}]{letter.py}
\lstinputlisting[language=Python, 
				breaklines=true,
				numbers=left,
				caption=frequency.py,
				label={frequency.py}]{frequency.py}
\chapter{Program Output}
\lstinputlisting[caption=Substring Count]{prob1_repeatcount.txt}
\lstinputlisting[caption=Full Ciphertext as Columns]{prob1_ciphertextcolumns.txt}
\lstinputlisting[caption=Column Freq. Count]{prob1_freqcount.txt}
\lstinputlisting[caption=Shift All Columns According to E]{prob1_columnshiftE.txt}
\lstinputlisting[caption=Full Successful Run]{prob1_full.txt}
\end{appendices}
\end{document}